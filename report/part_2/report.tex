\documentclass{article}

\usepackage{amsmath}
\usepackage{float}
\usepackage{csvsimple}
\usepackage{graphicx}
\usepackage[style=numeric, backend=bibtex8]{biblatex}

\graphicspath{ {../../results/} }

\addbibresource{citations.bib}

\begin{document}

  \title{Linear Time Coloring of Random Geometric Graphs}
  \author{Luke Wood}
  \maketitle

  \section{Executive Summary}
    \subsection{Introduction and Summary}
    \subsection{Programming Environment Description}
  		The implementation of the algorithm used to gather the data supporting this report was gathered on a 15 inch Macbook pro 2017 with a 2.9 GHz Intel Core i7 processor and 16 GB of RAM.
  		The computer is running macOS High Sierra.
  		The graph generation is written in python 3 as generating and connection a graph is not super computationally expensive with even decently large inputs such as 100000 (assuming $O(n) algorithms$).
  		The later algorithms may be implemented in a different language such as Elixir to get high levels of concurrency and higher efficiency due to type inference (as opposed to python's dynamic typing).
  \section{Reduction to Practice}
      This section will describe the transition from theory to implementation.
      This section will also give a detailed analysis of the algorithms used in this project as well as their asymptotic runtimes.
  	\subsection{Data Structure Design}
  	\subsection{Algorithm Description}
  	\subsection{Verification}
  \section{Result Summary}
  \printbibliography

\end{document}
